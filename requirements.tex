\documentclass{article}
\usepackage{enumitem}
\usepackage{titlesec}
\usepackage{titling}
\usepackage[margin=1in]{geometry}
\usepackage{underscore}

\titleformat{\section}
{\huge \bfseries}
{\thesection\ }
{0em}
{}[\titlerule]

\titleformat{\subsection}
{\LARGE \bfseries}
{\thesubsection\ }
{0em}
{}

\titleformat{\subsubsection}
{\Large \bfseries}
{\thesubsubsection\ }
{0em}
{}

\renewcommand{\maketitle}{
   \begin{center}
      {\Huge \bfseries Requirements Document}
   \end{center}
}

\newlist{steps}{enumerate}{1}
\setlist[steps, 1]{label = \underline{\hspace{2em}} Step \arabic*:}

\setlength{\parindent}{0em}

\begin{document}

\maketitle
\tableofcontents

\section{Introduction}
This document outlines the requirements for the ChocAn data processing software. This document will cover the product overview, the functional and non-functional requirements, and the milestones and deliverables that will be met along the development process.

\subsection{Purpose and Scope}
The purpose of this document is to inform stakeholders of the plan of progression of the ChocAn data processing software. It will also address the functional and non-functional requirements and deliverables.

\subsection{Target Audience}
The target audience for this document are the stake holders of the product. Anyone holding interest in the use or development of the ChocAn software can use this document to view the proposed process.

\subsection{Terms and Definitions}
Define any terms or acronyms you will be using in the remainder of this document

\section{Product Overview}
Give a high level description of the functionality of the project here. Describe the purpose of this section. It may be useful to give your definition of a user, a stake holder and a use case. If there are scope limitations to the project, i.e. things you will not be doing, or are not required to do, this is a good section to put those.

\subsection{Users and Stakeholders}
Describe the purpose of this section. Only a few sentences are expected here.
\subsubsection{Stakeholder 1}
List the first stakeholder or class of stakeholders if necessary. Describe, exactly, their role in the development, deployment, use, maintenance, etc. of the software.
\subsubsection{Stakeholder 2}

\subsection{Use Cases}
Describe the purpose of this section. Only a few sentences are expected here.

\subsubsection{Use Case 1}
Describe the first use case here. Be sure to explicitly identify the participants, human or otherwise, and explain their roles. Diagrams may be effective here, particularly a sequence diagram.

\subsubsection{Use Case 2}

\subsubsection{Use Case 3}

\section{Functional Requirements}
Describe the purpose of this section and outline its contents. Only a few sentences are expected here. It may help to define a functional requirement.

\subsection{Functional Requirement 1}
Describe the first functional requirement. This is the meat of the document, so be sure to use precise language. Include diagrams when appropriate.

\subsubsection{Functional Requirement 1 Apects 1}
It may be necessary to separate some of the larger functional requirements into several sub-requirements or requirement aspects.

\subsubsection{Functional Requirement 1 Apects 2}

\subsection{Functional Requirement 2}
Describe the second functional requirement. This is the meat of the document, so be sure to use precise language. Include diagrams when appropriate.

\subsubsection{Functional Requirement 2 Apects 1}
It may be necessary to separate some of the larger functional requirements into several sub-requirements or requirement aspects.

\subsubsection{Functional Requirement 2 Apects 2}

\section{Nonfunctional Requirements}
Describe the purpose of this section and outline its contents. Only a few sentences are expected here. It may help to define a nonfunctional requirement.

\subsection{Nonfunctional Requirement 1}
Describe the first Nonfunctional requirement. This is the meat of the document, so be sure to use precise language. Include diagrams when appropriate.

\subsubsection{Nonfunctional Requirement 1 Apects 1}
It may be necessary to separate some of the larger nonfunctional requirements into several sub-requirements or requirement aspects.

\subsubsection{Nonfunctional Requirement 1 Apects 2}

\subsection{Nonfunctional Requirement 2}
Describe the second nonfunctional requirement. This is the meat of the document, so be sure to use precise language. Include diagrams when appropriate.

\subsubsection{Nonfunctional Requirement 2 Apects 1}
It may be necessary to separate some of the larger nonfunctional requirements into several sub-requirements or requirement aspects.

\subsubsection{Nonfunctional Requirement 2 Apects 2}

\section{Milestones and Deliverables}
Describe the purpose of this section and outline its contents. Describe the milestones on a high level. Include a description of your workflow plan. A Gantt chart may be beneficial here.

\subsection{Milestone/Deliverable 1}
Explain what this milestone consists of. Identify the deliverable or deliverables that will be produced at this milestone. Generally describe what work will be done leading up to the conclusion of this milestone.

\subsubsection{Milestone/Deliverable 1 Stage 2}
Some of the larger milestones may be broken up into stages. Explain what is involved in this stage and what deliverable will be produced at the end of this stage.

\end{document}
