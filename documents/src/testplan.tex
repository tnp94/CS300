\documentclass{article}
\usepackage{hyperref}
\usepackage{enumitem}
\usepackage{titlesec}
\usepackage{titling}
\usepackage[margin=1in]{geometry}
\usepackage{underscore}
\usepackage{fancyhdr}
\pagestyle{fancy}
\fancyhead{}
\lhead{}
\chead{Austen Nelson --- Zhengmao Zhang --- Xiaoran Ge --- Xingjian Wang --- Michael Kay --- Thomas Pollard}
\rhead{}

\renewcommand{\maketitle}{
   \begin{center}
      {\Huge \bfseries Test Plan}
   \end{center}
}

\newlist{steps}{enumerate}{1}
\setlist[steps, 1]{label = \underline{\hspace{2em}} Step \arabic*:}

\setlength{\parindent}{0em}

\begin{document}

\maketitle
\tableofcontents
\pagebreak

\section{Introduction}
This test plan document will discuss the test methods used in the ChocAn
project. These testing methods include unit testing, smoke testing, and system
testing. After going through these tests, our team can discover the weaknesses
of this project and improve the project against these weaknesses.

\subsection{Purpose and Scope}
The purpose of this file is to test whether the ChocAn project can meet the
system requirements and whether it can meet the needs of users. In this test
plan, these goals need to be achieved:

\begin{itemize}
   \item Through unit testing, smoke testing, and system testing to ensure that the
      entire system is error-free and can meet the expected functional requirements.
   \item During the test, some improvements were made for some unsatisfactory parts.
   \item Improve understanding of testing.
\end{itemize}

\subsection{Target Audience}
The target audience of this test plan document is the maintenance team and test
team for this project. If anyone is interested in the testing process of this project,
you can also read this document. Before reading this document, you need to
understand the basic structure of the project. You can read the design document
or read the code of the document.

Relevent links:
\begin{itemize}
   \item \href{https://github.com/aujxn/CS300}{Project repository}
   \item \href{https://github.com/aujxn/CS300/documents/design.tex}{Design document}
\end{itemize}

\subsection{Terms and Definitions}
Here are some definitions that might be used:

\begin{itemize}
\item Unit Testing: UNIT TESTING is a level of software testing where
individual units/ components of a software are tested. The
purpose is to validate that each unit of the software performs
as designed. A unit is the smallest testable part of any
software. It usually has one or a few inputs and usually a single
output.
\item Smoke Testing: SMOKE TESTING, also known as “Build Verification Testing”, is
a type of software testing that includes of a non-exhaustive set of tests that aim
at ensuring that the most important functions work. The result of this testing is
used to decide if a build is stable enough to proceed with further testing.
\item System Testing: SYSTEM TESTING is a level of testing that validates the
complete and fully integrated software product. The purpose of a system test is
to evaluate the end-to-end system specifications.
\end{itemize}


\section{Test Plan Description}

\subsection{Scope of Testing}

\begin{itemize}
   \item Terminal test module \\
      Test whether the program can complete the basic operation in the
      computer terminal and reach the required menu as required.
   \item Maps Data Structure module \\
      Test whether the corresponding structure of the data storage
      location is implemented.
   \item User Hierarchy module \\
      Testing to ensure members, providers and managers have the right data
      and function.
   \item Service Record module \\
      Test to make sure a service is selected and recorded correctly as
      Providers and members.
\end{itemize}

\subsection{Testing Schedule}
The first test to be performed is a unit test. This will ensure that each sub-
component (unit) works properly and that all errors and incorrect outputs
found are fixed.

After unit testing is complete, the next testing phase will be smoke
testing. This will test the system as a whole, with each slave unit
interacting with all its systems and dependent units. Once the main
functions of the system as a whole are tested and completed, we will
transition to system testing.

The final testing phase will be system testing. This test will ensure that
the software meets the overall required specifications.

\subsection{Release Criteria}
Because the document involves the user's personal privacy, we must make
reasonable adjustments to the final released document.
\begin{itemize}
   \item Because the data contains sensitive information of members and
      providers For ChocAn, our final requirements will require the security of all
      our data and our data structure does not contain memory leaks that can
      be exploited.
   \item Our program should allow logins from different devices or terminals, and
      must identify different levels of personnel information.
   \item Allows the administrator to manage the services that are present in the
      system, and the details related to the services.
   \item Allow customers to get their own weekly reports, and display the current
      weekly report in real time when the user needs it.
\end{itemize}


\section{Unit Testing}
In the unit testing section, we will list the testing strategies and some functions
that need to be tested. In unit testing, we need to test the individual functions in
each object, and make sure that each object works correctly before linking them
together.

\subsection{Strategy}
In our system, we have built these classes:

\begin{itemize}
   \item InteractiveModule
   \item ManagerModule
   \item Member
   \item Person
   \item Provider
   \item ProviderDirectory
   \item ProviderModule
   \item Service
\end{itemize}

Some of these classes exist independently and have nothing to do with
other classes, so we decided to test these classes first. There are some
classes that need to interact with other classes. For these classes, we will
complete this part of the test after completing the previous part of the test.

\subsection{InteractiveModule}
InteractiveModule is implemented by relying on Member and Provider, so
InteractiveModule will be tested after testing Person, Member, and
Provider. InteractiveModule's private domain has two unordered_maps
that store information about Members and Providers, and public functions
mainly operate on these two unordered_maps, including add, remove,
edit, display. In the constructor, the data in the file is read and stored in
the two maps. In the write_out function, the data in the map is written to
the file. The init function will call other functions.

We will test the functions for the map operation and the functions of the
read and write parts, because these parts of the function are more likely to
make errors.

\subsection{ManagerModule}

The second object is the ManagerModule. This object depends on
Service, Member, Person, and Provider, so it will be tested later. This
object's private domain has three maps to store service, member, and
provider. Among its public functions, weekly_report and summary_report
are responsible for generating reports, and person_report,
provider_report, and member_report are responsible for sending reports to
specific people. Similarly, this object will read data in the constructor.

We will test the weekly_report function of this object and focus on seeing if
this function selects the date correctly. We will also test the person_report
function, because provider_report and member_report both call this
function, so we must ensure the correctness of this function.


\subsection{Person}

Person is a completely independent object, and it is also a subclass of
member and provider. Its private domain is name, city, state, id, and zip.
And its common functions are used to get this information and edit this
information.

This is a simple structured object, we only need to test whether the
functions can get and change the information stored in it.

\subsection{Member}
Member is a subclass of Person. Compared with person, member has one
more suspended variable. At the same time, get and set functions for
suspended variables have also been added.

We will test the get_suspended and set_suspended functions.

\subsection{Provider}
Provider is a subclass of person, without any new variables and functions,
so there is no need to test.

\subsection{Service}
Service is an independent class that is used to store service information
and provide external interfaces to access and modify the service.

We need to test whether the interfaces of these services are valid so that
no errors occur when calling these interfaces.

\subsection{ProviderDirectory}

ProviderDirectory has a map data structure that stores services, so it
needs to be tested after service unit testing. The common function of this
object is generate_directory (), which can write directory to directory.txt
file.

We are going to test the generate_directory () function to make sure the
format of the data meets our expectations.

\subsection{ProviderModule}

An instance of the providerdirectory object exists in the private domain of
the ProviderModule, so you need to test it after you finish testing the
providerdirectory. The public functions to be tested in this object are:
validate_member, find member by member id, and check suspended
status of member.
provide_service, output service information, and confirm that there is no
error in the service information.
get_provider_directory, write the service information to the directory.txt
file.

\section{Smoke Testing}
In the above, we tested the actual work of each function independently, and we
actually verified whether the function's return value is reasonable and effective.
And we verified the processing status of the data by a single function. In this
section, we will combine all the previous test results to perform linkage between
multiple functions to determine whether the cooperation between multiple
functions is effective.

\subsection{ManagerModule, ProviderModule, InteractiveModule Login}
\begin{tabular}{ |l|l|l| }
   \hline
   Case  &              Test           &     Expected Result          \\
   \hline
   \hline
   1     &    Login to manager module with valid ID      &   logs in    \\
   \hline
   2     &    Login to manager module with invalid ID    &   login is rejected  \\
   \hline
   3     &    Repeat 1 and 2 for InteractiveModule and ProviderModule &          \\
   \hline
\end{tabular}

\subsection{InteractiveModule Searching and Editing Members and Providers}
\begin{tabular}{ |l|l|l| }
   \hline
   Case  &              Test           &     Expected Result   \\
   \hline
   \hline
   1     &     Search for provider and member that exist   &   person is found    \\
   \hline
   2     &    Edit a provider and member that exist    &  data structure is updated correctly \\
   \hline
   3     &   Remove a provider and member that exist  &    data structure is updated correctly \\
   \hline
   4     &     Search for provider and member that doesn't exist   &   nothing is found    \\
   \hline
   5     &    Edit a provider and member that doesn't exist    &  data structure is unchanged and error is returned \\
   \hline
   6     &   Remove a provider and member that doesn't exist  &    data structure is unchanged and error is returned \\
   \hline
\end{tabular}

\subsection{Reports}
\begin{tabular}{ |l|l|l| }
   \hline
   Case  &              Test           &     Expected Result    \\
   \hline
   \hline
   1     &    Run provider report & correct data is reported  \\
   \hline
   2     &   Run member report &  correct data is reported  \\
   \hline
   3     &   Run weekly report & correct data is reported  \\
   \hline
\end{tabular}

\subsection{Provide Service}
\begin{tabular}{ |l|l|l| }
   \hline
   Case  &              Test           &     Expected Result      \\
   \hline
   \hline
   1     &   Provide service with valid info & service data structure is updated  \\
   \hline
   2     &  Proved service with invalid info & failure is returned and data structure is unchanged \\
   \hline
\end{tabular}


\section{System Testing}
In this part we go through three steps to test the integrity and functionality of the
program as expected. The three tests are: joint test, structural test, and
performance test.
\subsection{Joint Test}
In this test we test the complete program by combining all the units. The
main purpose of this test is to find out whether the link between the various
units and interfaces is smooth, and to troubleshoot some errors that are
only found at runtime.
\subsection{Permission Test}
In this test we will test whether the permissions of each type of user are
correct. For example, members can only change their basic information by
logging in to their account. Providers can only make changes to their
users; records of this service and cannot see the basic information and
other records of members.
\subsection{Storage test}
This test will check that the user information and records are stored
correctly and that the automatically sent abstracts and reports are properly
structured.
\subsection{Disk Recording}
It will include the following info:
\begin{itemize}
   \item Current date and time
   \item Date of service
   \item Provider's number
   \item Member number
   \item Service code
\end{itemize}

\subsection{Summary Report}
It will include the following info:
\begin{itemize}
   \item A list of providers to pay this week.
   \item Number of consultations per provider.
   \item The total cost to be paid by the provider during the week.
   \item Total number of providers providing services.
   \item Total number of consultations from all providers.
\end{itemize}
\subsection{Generated Form}
It will include the following info:
\begin{itemize}
\item Member name
\item Member number
\item Member address
\item Member cities
\item Member States
\item Member postal code
\end{itemize}
For each service provider, the following should exist:
\begin{itemize}
   \item Date of service
   \item Provider name
   \item service name
\end{itemize}
On the other hand, the provider's form should contain the following data:
\begin{itemize}
   \item Provider name
   \item Provider number
   \item Provider address
   \item Provider city
   \item Provider status
   \item he postal code of the provider
   \item Total number of consultations
   \item Total cost of the week
\end{itemize}
For each service, the following data:
\begin{itemize}
   \item Date of service
   \item The system receives date and time data
   \item Member name
   \item Member number
   \item Service code
   \item Fees to be paid
\end{itemize}

\section{Performance Test}
\subsection{Stress test}
In this test, the program's endurance limit is tested by entering a large
amount of data into the program at one time.
\subsection{Time test}
This test is to check whether the program produces results at the right
time for each input.
\subsection{Security Test}
In this test we will test the program's response by intentionally entering an
error code or performing an invalid operation.
\end{document}
